 % For referencing the chapter elsewhere, use \ref{Chapter1} 

\lhead{\cref{chapter:preface} \emph{Introduction}} % This is for the header on each page - perhaps a shortened title
\chapter{Preface}\label{chapter:preface}
\section{Introduction}
Image segmentation is the process of partitioning an image into sets of pixels, so called segments, that share a certain property. The goal is to optain a simple representation that is easier to analyse than the original image. Often, the objective is to detect the boundaries of an object in the image to examine it further. This has for example its applications in medical image analysis, where one might be interested in investigating the structure, shape and size of an organ, given for instance an MRI-scanning.\\

The levelset method is a flexible approach to do image segmentation. The idea is to evolve the contour that seperates the segments in the image. To accomplish this, a signed distance function is used, where the 0'th level curve represents the contour. The levelset function supports the propagation of the curve through the image.\\

In this project, Chan \& Vese's active contour method \cite{chan.01} utilizing a levelset function for 2-dimensional images is derived and described extensively in \cref{chapter:CVL}. The method is implemented using the programming details which are recorded in the appendix. The Chan Vese Levelset (CVL) method is stated in terms of a functional, based on measuring the similarity of two segments to their respective mean value. It also includes the regularizations for the contour length and the area of one segment in order to handle irregularities in an image.\\

The main subject of this project is to develop Orderless Levelsets (OL), a method based on the CVL formulation. The idea is to use the concept of Locally Orderless Images (LOI) to represent the image and the levelset function. In LOI, one discards the spatial information of the image and consider smooth histograms over the intensity values instead. The concept of LOI is described in \cref{chapter:LOI}.\\
This formulation leads to a robust formulation of the levelset method that is not dependent on solving differential equations, hence there are no risks for numerical instabilities or large equation systems to solve. Instead a uncomplicated functional is stated that can be solve using well known optimization techniques. The OL-method is thus a simple, flexible method that has no spatial dependencies. The spatial independence furthermore implies that the method can be generalized to $n$-dimensional images.\\
The Orderless Levelset method is developed, formulated and proven to be well defined in \sref{section:OLform}. In \sref{section:OLder}, the differentiability of the resulting functional is shown, which is necessary for the process of optimization. The formulation is concretized in \sref{section:OLopt} and the optimization process is described. Furthermore the method is implemented with choices that are documented in the appendix.\\

The functionality and the quality of the OL-method are verified and evaluated in \cref{chapter:experiments}. This is done by performing comparative experiments of the OL- and the CVL-method, covering a spectrum of different properties of the levelset methods. Moreover the computation times of CVL and OL are compared and the convergence is examined.\\
\cref{chapter:defthm} contains definitions and theorems necessary for the derivations used throughout the project. Finally, the report is concluded with an outlook on further work \cref{chapter:conclusion}.

\section{Previous Work}

Active contour models, often called snakes, are used to detect object boundaries in images by starting out with an inital curve. The idea is minimize a functional by evolving a curve through an image that reaches a minimum at the boundary of the object we are interested in.\\
Let $I:\Omega\to \R$ be an image defined on $\Omega\subset\R^2$ and let $C:[0,1]\to\R^2$ be a parameterized curve representing the contour.\\
Previous work on active contour models was done by Kass, Witkin and Terzopoulos \cite{kass.88}, where the energy functional

\begin{equation}
  E_{snake}(C) = \alpha\int_0^1 \abs{C'(s)}^2  + \beta\int_0^1 \abs{C''(s)}^2\df s + \lambda\int_0^1 E_{image}
\end{equation}

is minimized. Here, $\alpha, \beta, \lambda$ are positive scaling parameters. The first term is the so called internal energy-term that controls the smoothness of the contour. The second term represents the image forces, pulling the contour towards the object in the image. Three different formulations for $E_{image}$ were posed: a line functional given by the intensity of the image, a edge detector functional given by $-\abs{\nabla I(C(s))}^2$ that attracts the curve to contours with large image gradients, and a smoothened version of the edge detector and the line functional using a Gaussian kernel. Active contour models using an edge detector rely on edges. When considering smooth images without clear contours this class of methods might have problems converging properly. Furthermore, local minima of the functional can be fatal to the evolution of the method.\\

To consider the evolution in a global sense, the levelset method was introduced by Osher \& Sethian \cite{osher.88} for motion of propagating surfaces under curvature. This concept has many applications, among other things for active contour models. The curve $C$ that is evolved through the image is represented by a Lipschitz function $\phi$ such that $C=\{(x,y)\Omega\;|\; \phi(x,y) = 0\}$, $\phi(x,y)> 0$ inside $C$ and $\phi(x,y)<0$ outside $C$. Introducing a time step $t$, the curve is initialised at $\phi(x,y,0)$ and evolved in the normal direction of the curve. The advantage that the levelset function lives on the entire domain, hence it does not necessarily stop at a local minimum, where an edge appears in the image.\\

In image segmentation, Mumford \& Shah developed general theory in \cite{mumford.89}. They consider the decomposition of the image domain $\Omega$ into disjoint subdomains $\Omega_i$ covering $\Omega$, in which the image varies smoothly and/or slowly. This leads to the formulation the following functional for segmentation:

\begin{equation}
  F(C,f) = \mu\cdot Length(C) + \lambda \int_\Omega \abs{I-f}^2 \df x\df y + \int_{\Omega\backslash C} \abs{\nabla f}^2 \df x\df y,
\end{equation}

where $\mu,\lambda$ are positive parameters and $f$ is a piecewise differentiable function approximating the image on the disjoint subdomains $\Omega$. The first term in $F$ assures that the boundary of the segments is kept as small as possible. The second term assures that the difference between $f$ and $I$ is kept as small as possible, hence $f$ should approximate $I$ on each $\Omega_i$. The third term assures that $f$ and thus $I$ does not vary much on $\Omega_i$.\\
This leads the active contour model by Chan \& Vese \cite{chan.01} using the levelset method. The method uses a variant of the Mumford \& Shah functional choosing $f$ to be represented by the average of the intensity values in the two segments. The method is elaborated in \cref{chapter:CVL}.\\

In order to develop the Orderless Levelset theory, the Locally Orderless Images framework is essential. It introduced in \cite{griffin.97, koe.99}. This framework suggests looking at local histograms and discarding the global spatial structure in images and is presented in detail in \cref{chapter:loi}.\\
This framework was applied to image registration by Darkner \& Sporring in \cite{dar.11,dar.12} where the target is measure and minimize similarities of a reference image $J$ and a image $I$ by transforming $I$. The similarity measure resembles the formulation of the target in this project - the Orderless Levelsets method. The article also includes a thouroughly analysis of the relation between the scales of Locally Orderless Images.